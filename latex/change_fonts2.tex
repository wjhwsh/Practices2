%!TEX TS-program = xelatex
%!TEX encoding = UTF-8 Unicode

% XeLaTeX 測試範例
% 以上是設定使用 XeLaTeX 編譯,並內設 unicode 編碼,以便 TeXShop 自動辨識

\documentclass[12pt]{article}

\usepackage{fontspec}
% XeTeX 配合 fontspec 時,字形非常容易設定 

\setromanfont[Mapping=tex-text, %
Ligatures={Required,Common}, %
ItalicFont={Times Italic}, %
BoldFont={Apple LiGothic Medium}]%
{BiauKai}
% 預設字形:這裡設內文為標楷,
% 沿用 latex 的一些標點的轉換,如 en-dash 以兩個減號表示,
% 如果此字型檔裡有 ligatures 的定義,則啟動
% 斜體字以 Times Iatlic (只有英文有斜體)
% 粗體字以黑體字表現
% 此為 fontspec 套件的指令

\setmonofont[Scale=0.8]{AppleGothic Regular} 
% 設定等寬字形

\XeTeXlinebreaklocale "zh" 
\XeTeXlinebreakskip = 0pt plus 1pt 
% 讓中文正確斷行的設定

\newfontfamily\rmfont{Times}
\newcommand{\nc}[1]{{\rmfont #1}}
% 在中文裡用英文字體顯示英文的命令
% 詳見底下的範例

\begin{document}

自此,就是內文了。可以自由的使用。

我們再來試試換預設字形的方法:只在大括號內局部之字串有效。

{
\fontspec{Hiragino Mincho Pro W6}
% \fontspec 用法和前面預設字形之 \setromanfont 一樣,只是可更自由地使用
這是用日文字形的測試。只要打正常中文即可。可能會缺字就是了。
}

再來就回到正常字形,就用正常 \LaTeX\ 的寫法使用就行了。

如果,在中、英文夾雜時,英文想用不同的字型,則可以使用開頭時定義的 nc 指令

\nc{This is Times font.  The 『field' contains ligatures.} 

應該與中文字型裡的英文字不同:
This is Chinese font. The `field' does not have ligatures.

\end{document}
